この論文では,OpenFOAM\textregisteredに実装されている2つの数値計算手法を,水中のサブミリメートル気泡の界面分解シミュレーションへの適合性について評価した.
コードinterFoamは,標準的な連続体表面力と曲率計算を備えた,名目上シャープな界面に対する代数流体体積法を実装している.
一方,phaseField-Foamは,自由エネルギー定式化における毛細管項を備えた拡散界面表現による位相場法に基づいている.
重力のない静止液体に囲まれた静的な円形気泡をテストケースとした.
二次元シミュレーションでは,一様格子のメッシュサイズ(h)と気泡サイズの変化による影響を調べた.
気泡形状,質量保存,ヤング-ラプラス圧力ジャンプ,スプリアスの大きさを評価することで,ソルバーの精度を評価した.

密度/粘度を一致させ,人為的に大きな非粘性落差を与えるという単純化された条件の2つのテストケースでソルバーを検証することにより,文献の調査結果を参照することができる.
コードinterFoamでは,スプリアスがメッシュ解像度によって減少することはなく,2次収束で減少するphaseFieldFoamよりも数桁大きいことが明らかになった.
この収束は,曲率計算がないこと,界面の厚さがメッシュ分解能に依存せず毛細管幅(\epsilon )によって制御されること,圧力勾配と毛細管項の間のバランスのとれた離散化が容易に達成できることに起因している.

静止水中のサブミリメートルの気泡という数値的に困難なケースでは,スプリアスはメッシュサイズに依存せず,phaseField-Foamの2次収束を失う.
しかし,スプリアスの大きさは,interFoamより5桁小さいままである.
interFoamの大きな非対称スプリアスは,特に形状が振動している高分解能においてバブルを大きく変位させ変形させるが,phaseFieldFoamではバブルは常に円形のままで初期位置を保ちます.
後者のコードは,ラプラス圧力ジャンプを1\% 以下の誤差で予測するが,前者は約13\% 過小評価する.
コードinterFoamは,細かい格子で0.1\% 以下の質量保存の非常に低い誤差を示すのに対して,phaseFieldFoamはシミュレーションの過程で気泡を収縮させ,秩序パラメータを大域的に保存する.

